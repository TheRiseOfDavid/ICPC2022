\section{基礎}
\subsection{關鍵字思考}
\raggedbottom\lstinputlisting[style=txt]{code/keyword.txt}
\hrulefill
\subsection{C++ 基礎}
\raggedbottom\lstinputlisting[style=cpp]{code/C++_basic.cc}
\hrulefill
\subsection{C++ 易忘的內建函數}
\raggedbottom\lstinputlisting[style=txt]{code/C++_remind.txt}
\hrulefill
\subsection{python 常用}
\raggedbottom\lstinputlisting[style=py]{code/python.py}
\hrulefill

\section{承恩-數論}
\subsection{約瑟夫斯-每兩個殺一次}
\raggedbottom\lstinputlisting[style=cpp]{code/two_Josephus.cc}
\hrulefill
\subsection{約瑟夫斯-一般情況}
\raggedbottom\lstinputlisting[style=cpp]{code/Josephus.cc}
\hrulefill
\subsection{最大子數列}
\raggedbottom\lstinputlisting[style=cpp]{code/kadane.cc}
\hrulefill

\section{建榮-演算法}
\subsection{深度優先搜尋}
\raggedbottom\lstinputlisting[style=cpp]{code/DFS.cc}
\hrulefill
\subsection{廣度優先搜尋}
\raggedbottom\lstinputlisting[style=cpp]{code/BFS.cc}
\hrulefill
\subsection{二分搜}
\raggedbottom\lstinputlisting[style=cpp]{code/Binary_Search.cc}
\hrulefill
\subsection{二元數的走訪}
\raggedbottom\lstinputlisting[style=cpp]{code/Binary_Tree.cc}
\hrulefill
\subsection{最大公因數}
\raggedbottom\lstinputlisting[style=cpp]{code/GCD.cc}
\hrulefill

\section{建榮-幾何}
\subsection{高中數學}
\raggedbottom\lstinputlisting[style=cpp]{code/math_of_highschool.cc}
\hrulefill
\subsection{點的模板}
\raggedbottom\lstinputlisting[style=cpp]{code/Point.cc}
\hrulefill
\subsection{向量計算}
\raggedbottom\lstinputlisting[style=cpp]{code/vector.cc}
\hrulefill
\subsection{直線模板}
\raggedbottom\lstinputlisting[style=cpp]{code/Line.cc}
\hrulefill
\subsection{找三角形外心}
\raggedbottom\lstinputlisting[style=cpp]{code/circumcenter.cc}
\hrulefill
\subsection{點在直線的上或下}
\raggedbottom\lstinputlisting[style=cpp]{code/intersection_line_point.cc}
\hrulefill
\subsection{兩直線交點}
\raggedbottom\lstinputlisting[style=cpp]{code/intersection.cc}
\hrulefill

\section{大衛-動態規劃}
\subsection{背包問題}
\raggedbottom\lstinputlisting[style=cpp]{code/01_Knapsack_Problem.cc}
\hrulefill
\subsection{LCS}
\raggedbottom\lstinputlisting[style=cpp]{code/LCS_ologn.cc}
\hrulefill
\subsection{LIS}
\raggedbottom\lstinputlisting[style=cpp]{code/LIS.cc}
\hrulefill
\subsection{Directed Acyclic Graph}
\raggedbottom\lstinputlisting[style=txt]{code/DAG.txt}
\hrulefill

\section{大衛-圖論}
\subsection{歐拉回路}
\raggedbottom\lstinputlisting[style=cpp]{code/euler.cc}
\hrulefill
\subsection{floyd 最短路徑}
\raggedbottom\lstinputlisting[style=cpp]{code/floyd.cc}
\hrulefill
\subsection{最小生成樹}
\raggedbottom\lstinputlisting[style=cpp]{code/kruskal.cc}
\hrulefill
\subsection{找圖中的橋 find bridge}
\raggedbottom\lstinputlisting[style=cpp]{code/graph_theorm_find_bridge.cc}
\hrulefill
\subsection{拓樸排序}
\raggedbottom\lstinputlisting[style=cpp]{code/Topological_Ordering.cc}
\hrulefill
\subsection{Component Kosaraju's Algorithm 找出 SCC}
\raggedbottom\lstinputlisting[style=cpp]{code/Component_Kosaraju_Algorithm.cc}
\hrulefill
\subsection{dijkstra}
\raggedbottom\lstinputlisting[style=cpp]{code/dijkstra.cc}
\hrulefill
\subsection{二分匹配、二分圖}
\raggedbottom\lstinputlisting[style=cpp]{code/Maximum_Bipartite_Matching.cc}
\hrulefill

\section{大衛-資料結構}
\subsection{並查集}
\raggedbottom\lstinputlisting[style=cpp]{code/Disjoint_Set.cc}
\hrulefill
\subsection{線段樹}
\raggedbottom\lstinputlisting[style=cpp]{code/segment_tree.cc}
\hrulefill

\section{大衛-字串}
\subsection{KMP}
\raggedbottom\lstinputlisting[style=cpp]{code/KMP.cc}
\hrulefill
\subsection{最短修改距離}
\raggedbottom\lstinputlisting[style=cpp]{code/Minimum_Edit_Distance.cc}
\hrulefill

